\section*{Método de Investigação}
A metodologia utilizada para realização do presente trabalho foi dividida da seguinte forma:
\begin{itemize}
	\item Pesquisa bibliográfica para obter o embasamento teórico sobre funcionamento dos sistemas envolvidos.
	\item Identificação de uma possível forma de integração entre ambos.
	\item Desenvolvimento da integração entre os sistemas.
	\item Desenvolvimento de uma suíte de testes automatizados.
	\item Experimentação utilizando a suíte de testes automatizados.
\end{itemize}

\section*{Estruturação da Monografia}
	
Após este capítulo introdutório, que basicamente visa contextualizar e caracterizar o tema de pesquisa, o trabalho realizado foi dividido em seis capítulos, conforme descrito abaixo:
\begin{description}
	\item \textbf{Capítulo 2} – Fundamentação Teórica – Este capítulo tem como objetivo abordar alguns dos conceitos do saneamento brasileiro e como o sistema GSAN está construído para atendê-lo, quais os módulos que compõem o sistema e detalhar a arquitetura do sistema, assim como expor os conceitos que envolvem o software Asterisk.
	\item \textbf{Capítulo 3 } – Revisão Bibliográfica – Será apresentado os principais conceitos utilizados como base no desenvolvimento deste trabalho.
	\item \textbf{Capítulo  4} – Middleware: GSAN e Asterisk – Trata-se da implementação realizada para integração entre os sistemas, apresentando as principais etapas para elaboração da comunicação entre os sistemas.
	\item \textbf{Capítulo  5} – Experimentação Utilizando a Suíte de Testes Automatizados – Tem como característica a preparação do ambiente de teste, o planejamento e execução das experimentações utilizando a suíte de testes automatizados para validar a integração entre os sistemas GSAN e Asterisk.
	\item \textbf{Capítulo 6} – Considerações Finais e Trabalhos Futuros – Finalmente, no sétimo capítulo, apresentam-se a conclusão que foi obtida e as recomendações para trabalhos futuros reunindo os comentários finais deste trabalho de pesquisa.	
\end{description}
	

\section*{Cronograma}
A seguir será apresentado o cronograma com o planejamento mensal do início e termino das atividades previstas para conclusão do trabalho de pesquisa.

\begin{table}[htb]
	\center
	\footnotesize
	\begin{tabular}{|p{5cm}|p{0.8cm}|p{0.8cm}|p{0.8cm}|p{0.8cm}|p{0.8cm}|p{0.8cm}|p{0.8cm}|p{0.8cm}|}
		\hline
		\centering
		\textbf{Atividade} & \textbf{Jun} & \textbf{Jul} & \textbf{Ago} & \textbf{Set} & \textbf{Out} & \textbf{Nov} & \textbf{Dez}\\
		\hline
		Pesquisa bibliográfica. 			& \textbullet & \textbullet & - & - & - & - & - \\
		\hline
		Identificar formas de integração.	& - & \textbullet & \textbullet & - & - & - & - \\
		\hline
		Implementar a integração. 			& - & - & \textbullet & \textbullet & \textbullet & - & -  \\
		\hline
		Realizar experimentos. 				& - & - & - & - & \textbullet & \textbullet & - \\
		\hline
		Escrita da monografia. 				& \textbullet & \textbullet & \textbullet & \textbullet & \textbullet & \textbullet & \textbullet \\
		\hline
		Apresentação da monografia. 		& - & - & - & - & - & - & \textbullet \\		
		\hline
	\end{tabular}
\end{table}

