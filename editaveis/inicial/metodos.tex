\section{Metodo de Investigação}
A metodologia a utilizada para realização do presente trabalho foi divida da seguinte forma:
\begin{itemize}
	\item Pesquisa bibliográfica para obter o embasamento teórico sobre funcionamento dos sistemas envolvidos
	\item Identificação das possíveis formas de integração entre ambos;
	\item Desenvolvimento da integração entre os sistemas;
	\item Realizar experimentações em ambiente isolado com cenários representando o atendimento de serviços de saneamento; 
	\item Avaliação dos resultados obtidos nos experimentos quanto à redução da taxa de atendimento realizado por atendentes.
\end{itemize}

\section{Estruturação da Monografia}
	
Após este capítulo introdutório, que basicamente visa contextualizar e caracterizar o tema de pesquisa, o trabalho realizado foi dividido em sete capítulos descritos, conforme descrito abaixo:
\begin{description}
	\item \textbf{Capítulo 2} – Fundamentação Teórica – Este capítulo tem como objetivo abordar alguns dos conceitos do saneamento brasileiro e como o sistema GSAN está construído para atendê-lo, quais os módulos que compõem o sistema e detalhar a arquitetura do sistema, assim como expor os conceitos que envolvem o software Asterisk.
	\item \textbf{Capítulo 3 } – Revisão Bibliográfica – Será apresentado os principais conceitos utilizados como base no desenvolvimento deste trabalho.
	\item \textbf{Capítulo  4} – Processo de Integração – Trata-se da implementação realizada para integração entre os sistemas, apresentando as principais etapas para elaboração da comunicação entre os sistemas.
	\item \textbf{Capítulo  5} – Resultados Alcançados – Tem como característica a realização das experimentações e descrição dos resultado obtidos como resultado da integração entre o sistema GSAN com o \textit{software} Asterisk.
	\item \textbf{Capítulo 6} – Considerações Finais e Trabalhos Futuros – Finalmente, no quinto capítulo, apresentam-se a conclusão e recomendações para trabalhos futuros, reunindo os comentários finais deste trabalho de pesquisa.	
\end{description}
	





\section*{Cronograma}
A seguir será apresentado o cronograma com o planejamento mensal do início e termino das atividades previstas para conclusão do trabalho de pesquisa.

\begin{table}[htb]
	\center
	\footnotesize
	\begin{tabular}{|p{5cm}|p{0.8cm}|p{0.8cm}|p{0.8cm}|p{0.8cm}|p{0.8cm}|p{0.8cm}|p{0.8cm}|p{0.8cm}|}
		\hline
		\centering
		\textbf{Atividade} & \textbf{Jan} & \textbf{Fev} & \textbf{Mar} & \textbf{Abr} & \textbf{Mai} & \textbf{Jun} & \textbf{Jul} & \textbf{Ago} \\
		\hline
		Pesquisa bibliográfica. 			& \textbullet & \textbullet & - & - & - & - & - & - \\
		\hline
		Identificar formas de integração.	& - & \textbullet & \textbullet & - & - & - & - & - \\
		\hline
		Implementar a integração. 			& - & - & \textbullet & \textbullet & \textbullet & - & - & - \\
		\hline
		Realizar experimentos. 				& - & - & - & - & \textbullet & \textbullet & \textbullet & - \\
		\hline
		Escrita da monografia. 				& \textbullet & \textbullet & \textbullet & \textbullet & \textbullet & \textbullet & \textbullet & \textbullet \\
		\hline
		Apresentação da Monografia. 		& - & - & - & - & - & - & - & \textbullet \\		
		\hline
	\end{tabular}
\end{table}

