\section{Aspectos de Inovação}
O trabalho de pesquisa e desenvolvimento se trata de uma integração entre software totalmente distintos, com tecnologia Open Source, onde juntos seram capazes de atender à uma demanda existente no setor de saneamento brasileiro relacionado ao contexto de Atendimento ao Público. Com a integração entre o sistema GSAN com o software Asterisk, será possível transferir os atendimentos destinado a central de atendimento, para uma Unidade de Resposta Audível (URA), executando a triagem das solicitações de forma padronizada e para solicitações referentes aos tipos de serviços Obter 2ª via de Conta, Informar Falta de Àgua e Solicitar Restabelecimento de Ligação, será possível realizar de forma automatizada o atendimento, sem que haja intervenção humana durante o processo. 
Não há oficialmente uma versão publicada na comunidade do sistema GSAN capaz de atender esta demanda de forma igual ou superior, inovando em propor uma solução viável e eficiente de baixo custo, para possibilitando a melhoria no atendimento ao cliente, se destacando em permitir que o cliente possa realizar suas solicitações em qualquer horário do dia ou noite, sem ter que se locomover a empresa e acessar informações de débitos pendentes de forma rápida e padronizada.
