\chapter[Introdução]{\textbf{\uppercase{Introdução}}}
%\addcontentsline{toc}{chapter}{Introdução}

A Secretaria Nacional de Saneamento Ambiental (SNSA) que integra o Ministério das Cidades (MC) visando elevar o nível de desempenho e eficiência das empresas de abastecimento de água e coleta de esgoto teve a iniciativa de promover o desenvolvimento de um software que pudesse atender as necessidades básicas do setor de saneamento de um modo geral.

Por meio do Programa de Modernização do Setor de Saneamento \cite{PMSS:2014} efetuou a contratação de uma empresa de tecnologia da informação brasileira para executar o projeto concebido.  Nesse cenário, surge então o sistema GSAN\footnote{GSAN - Sistema Integrado de Gestão de Serviços de Saneamento.} que se trata de um sistema desenvolvido com tecnologias de software livre, para a gerência de operações comerciais e de controle da execução de serviços internos das companhias de saneamento, o software atualmente encontra-se disponível gratuitamente no portal do software público brasileiro \cite{PORTAL:2014}. 

Mesmo com a modernização do setor de saneamento através do sistema GSAN que atualmente está implantado em 10 companhias estaduais das quais 4 estão em processo de migração \cite{PMSS:2014}, ainda há grandes desafios a serem superados e um deles será abordado neste trabalho.

O Atendimento ao público trata-se de uma das frentes que as empresas de saneamento necessitam disponibilizar aos seus clientes. Muitas das vezes o valor envolvido em manter disponível uma infraestrutura que atenda a necessidade da empresa, com equipes de \textit{Call Center} ou mesmo com atendimento presencial, podem gerar custos astronômicos dependendo da quantidade e qualidade da mão de obra contratada, aquisição de licenças para soluções proprietárias entre outros fatores que podem contribuir para variação do valor. 

Atualmente o uso de software \textit{Open Source} nas empresas tem se tornado bastante comum \cite{MEIRELLES2014}, com intuito de apoiar o negócio, como é o caso do software Asterisk que implementa facilidades no uso de tecnologias como PABX\footnote{PABX - \textit{Private Automatic Branch Exchange.}}  tanto para linhas telefônicas convencionais como também por VoIP\footnote{VoIP - \textit{Voice over Internet Protocol}} que utiliza a transmissão de voz sobre um rede IP\footnote{IP - \textit{Internet Protocol}} com padrão de qualidade de serviço (QoS), permitindo a utilização de URA (Unidade de Resposta Audível) (Vieira, 2007) como linha de frente no atendimento ao cliente.


