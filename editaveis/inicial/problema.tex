\section*{PROBLEMA}
%\vspace{1.5cm}
Atualmente o GSAN, atende grande parte das companhias de saneamento brasileiras, como é o caso de companhias como por exemplo CAERN\footnote{CAERN - Companhia de Água e Esgotos do Rio Grande do Norte}, CAER\footnote{CAER - Companhia de Água e Esgotos de Roraima}, COMPESA\footnote{COMPESA - Companhia Pernambucana de Saneamento}, MANAUS AMBIENTAL entre outras citadas no seguinte referencial teórico \ref{key:GSAN-TEORIA}.

Essas empresas fazem uso do sistema para gerenciar as suas informações operacionais e gerenciais, que de certa forma atende as demandas internas. Entretanto, no aspecto do atendimento público, existem lacunas que ainda precisam ser atendidas de forma plena, principalmente por se tratarem de organizações altamente demandadas pela população, sendo responsáveis por atender diversos tipos de clientes que variam desde pequenos vilarejos até grandes metrópoles.

A falta de padronização nos atendimentos, o grande fluxo de transferência entre ramais e a variação nos tempos de atendimentos são muito comuns, pelo fato de todo atendimento de Primeiro Nível\footnote{Atendimento de Primeiro Nível Refere-se a recepção inicial de todo atendimento.} normalmente ser realizado por pessoas ou PA (Posto de Atendimento), ou seja, são recursos caros. Normalmente os atendentes respondem por um determinado setor da empresa, encarregado em solucionar tipos específicos de problemas, possibilitando muita das vezes a realização de transferência para outros ramais até que o cliente consiga concluir uma solicitação, o que pode gerar desconforto e insatisfação com os serviços de atendimento.
	
Existe uma grande dificuldade das empresas de saneamento, em disponibilizar uma estrutura de \textit{Call Center} que atenda as expectativas dos clientes.

A dificuldade em manter um \textit{feedback} rápido com o cliente, a falta de canais de comunicação flexíveis que permitam uma disponibilidade maior inclusive fora do horário comercial, a ineficiência na triagem dos atendimentos, são questões rotineiras enfrentadas no cotidiano das empresas de saneamento.
