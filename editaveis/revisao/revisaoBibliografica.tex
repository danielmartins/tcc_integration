\chapter[Revisão Bibliográfica]{\textbf{R}evisão \textbf{B}ibliográfica}
\addcontentsline{toc}{chapter}{Revisão Bibliográfica}

\textit{Neste capítulo será apresentado o conceito dos frameworks utilizados neste trabalho, além de justificar o motivo que levou a ser utilizados e destacar vantagens obtidas em seu uso.}


\section{Tecnologias Utilizadas}

No estudo para propor uma solução viável e consistente de integração entre sistemas, que seja realmente eficiente, é necessário entender todo o contexto em que está sendo operado o Sistema de Informação GSAN, visando identificar as informações mais relevantes acessadas pelo atendimento ao cliente, as principais dificuldades enfrentadas e os desafios que norteiam esse módulo do sistema. 


\subsection{GSAN - Detalhamento Técnico }

O GSAN por ser um sistema de informação adaptável a empresas de pequeno, médio e grande porte, contemplando soluções dos mais diversos requisitos, entre eles Cadastramento, Micromedição, Faturamento, Arrecadação, Cobrança, Negativação e Atendimento ao Cliente. Fornecendo de forma razoavelmente flexível as configurações e detalhes operacionais das rotinas, disponível em ambiente Web utilizando recursos de tecnologias software livre.

Para realização de melhorias no sistema GSAN faz-se necessário o ter conhecimento sobre os principais frameworks utilizados, a linguagem de programação utilizada e alguns conceitos de Saneamento que serão abordados.
Dos principais Frameworks utilizados, destaco o uso dos seguintes: \\

\textbf{Hibernate}: Trata-se de um robusto framework de persistência de objetos relacionais, que fornece facilmente meios para realizar o mapeamento das entidades do sistema e diminui a complexidade de acesso a base de dados. (Hibernate, 2014). \\

\textbf{Apache Struts}: Tem como característica principal a sua utilização na construção de modernos controladores utilizando o padrão \textit{Model View Control} (MVC) que se trata da separação das camadas utilizadas em uma aplicação, fornecendo uma maior organização no código fonte e contribui para futuras manutenções. (FOWLER, 2015). \cite{APACHE_STRUTS}

O sistema GSAN faz uso da plataforma Java, lançado na versão Java Develop Kit (JDK) 1.5, utilizando recursos especificados pela \textit{Java Enterprise Edition} (JEE) (PORTAL, 2014), essencialmente o container \textit{Enterprise Java Bean} (EJB), \textit{Java Server Pages }(JSP) e \textit{Servlets} que são executadas dentro de um servidor de aplicação Java EE.


\subsection{Asterisk - Detalhamento Técnico}
A ferramenta de código aberto Asterisk, tem algumas características importantes e fundamentais para o estudo além de ser uma implementação de uma central telefônica que permite que clientes se comuniquem, tem outros recursos interessantes que fazem da ferramenta uma peça chave no processo da integração proposta, recursos como respostas interativas, correios de voz, realização de conferencias, distribuição automática de chamadas \cite{ASTERISK}, além de ser flexível a adição de novos recursos tanto por meio de scripts na própria linguagem do Asterisk como também por meio de códigos em linguagem C entre outras formas de customização da ferramenta. Desenvolvido pela empresa Digium sob licença GPL (\textit{General Public Lisence}), atualmente portável em versões Linux, Windows e Mac OS, suportando protocolos de Voz sobre IP (VoIP), assim como SIP e H.323 entre outros. O próprio Asterisk contem um protocolo próprio chamado IAX fornecendo um melhor desempenho entre os entroncamentos entre os servidores Asterisk para casos de maior complexidade.

\subsection{Simple Object Access Protocol}
O protocolo SOAP (Simple Object Access Protocol) tem o objetivo de possibilitar a troca de informações estruturadas em Linguagem de Marcação Extensível (XML), para sistemas distribuídos. A negociação e transmissão de mensagens foram baseadas em outros em outros protocolos de serviços como o HTTP (Hypertext Transfer Protocol) e RPC (Remote Procedure Call), possibilitando a utilização para realizar integrações entre softwares.

\subsection{Asterisk Gateway Interface}
Um dos protolocos de interface de comunicação  com o software Asterisk chamado AGI (\textit{Asterisk Gateway Interface}), tem o objetivo de possibilitar que o software consiga consumir recursos externos através de requisição semelhantes ao CGI(\textit{Common Gateway Interface}) de servidores web, no entanto de forma inversa as requisições são originadas pelo próprio Asterisk (Asterisk-Java, 2015).

\subsection{WebServices}
O WebService se trata de uma solução que permite que sistemas diferentes se comuniquem através requisições de protoloco HTTP a recursos identificados por um URI (\textit{Uniform Resource Identifier}) identifico a Web convencional, descritos e definidos usando XML (\textit{Extensible Markup Language}) (WebService, 2015).

\subsection{Unidade de Resposta Audível}
A Unidade de Resposta Audível (URA) ou atendente eletrônico se trata de um software ou equipamento de \textit{Call Center}, que possibilita o atendimento das ligações de forma automática, tal solução traz como benefício à padronização dos atendimentos e tem potencial para automatização dos atendimentos, com inúmeras possibilidades de customização através de integrações com sistemas externo (Asterisk, 2014) e (VIEIRA, 2007).

\subsection{Middleware}
O Middleware ou intermediário se trata de uma camada de software responsável em mediar à comunicação de outros sistemas, utilizado normalmente em ambientes que tendem a utilizar plataformas, linguagens ou protocolos de comunicação diferentes nas trocas de informação, sendo um dos recursos adotados neste trabalho, tal recurso descrito por \citeonline{ALMEIDA2011}.

\subsection{Disc-OS}
O Disc-OS refere-se a uma distribuição Linux (Cent-OS), customizada para utilização de PABX e PABX IP, abstrai toda a configuração de bibliotecas básicas e instalação do Asterisk (Disc-OS, 2015). Disponibiliza uma interface Web para configuração dos recursos, são algumas das vantagens, possuir disponível no idioma português brasileiro, além de tornar bem prático o processo de configurações.

\subsection{Codec}
O codec (COder/DEcoder) se trata de processo de codificação e decodificação da voz humana a ser transmitidas entre a origem e destino em meio digital (VIEIRA, 2007).