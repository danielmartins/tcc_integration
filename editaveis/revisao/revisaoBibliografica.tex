%\chapter[Revisão Bibliográfica]{\textbf{R}evisão \textbf{B}ibliográfica}
%\addcontentsline{toc}{chapter}{Revisão Bibliográfica}

%\textit{Neste capítulo será apresentado o conceito dos \textit{frameworks} utilizados %neste trabalho, além de justificar o motivo que levou a ser utilizados e destacar %vantagens obtidas em seu uso.}

\section*{Trabalhos Relacionados}
Este trabalho de pesquisa e desenvolvimento se assemelha ao trabalho descrito por Guilherme \cite{VIEIRA:2007}, que também utilizou recursos do programa Asterisk para desenvolver uma Sistema de criação de planos de discagem de forma prática explanando aspectos da ferramenta e expondo as dificuldades encontradas, apesar de ambos utilizarem dos diversos recursos do Asterisk, há divergência no objetivo onde este se destaca o fato de realizar uma integração com outro software, visando solucionar uma demanda do setor de Saneamento.

No trabalho desenvolvido por Jilcimaico \cite{DARU:2008}, aborda com clareza a utilização da distribuição Disc-OS como interface WEB do Asterisk, além de descrever os principais conceitos envolvidos na utilização do software, demonstra os procedimentos necessários para realizar a instalação da ferramenta e configuração dos recursos essências para um \textit{Call Center}, assemelhando-se este ao fato de também utilizar a distribuição Disc-OS que propõe uma interface WEB para a configuração do Asterisk.

O trabalho desenvolvimento por Humberto \cite{CAMPOS:2007}, utilizou os principais recursos do software Asterisk para realizar uma integração com um sistema externo que calcula os valores de cada ligação realizada, módulo chamado de “tarifador” além de exibir os valores em um hardware próprio, tal integração utilizou como referência tabelas em banco de dados para reconhecer eventos ocorridos e ações a serem tomadas, assemelhando-se a este trabalho pelo fato de utilizar recursos do Asterisk para disparar ações de sistemas externos, no entanto a forma de integração retratada acima se diferencia da forma adotada neste trabalho, que utiliza a interface AGI disponibilizada para comunicação com sistemas externos, o próprio Asterisk irá disparar ações a serem realizadas por meio de um \textit{Middleware} que implementa esta interface de comunicação.

Atualmente a empresa de saneamento Companhia Pernambucana de Saneamento \cite{COMPESA:URA} disponibilizou aos seus clientes o atendimento eletrônico por meio de URA, possibilitando a empresa realizar o atendimento destinado a central de atendimento, ou seja, o atendimento de primeiro nível, de forma automática e padronizada, propiciando também os direcionamentos entre ramais reais da empresa agilizando o atendimento e potencializando uma disponibilidade de 24 horas por 7 dias, com as informações à disposição dos clientes remotamente, porém a empresa não divulgou detalhes técnicos ou artefatos produzidos para realizar tal integração ou customização.
Para auxílio na elaboração deste trabalho de pesquisa se fez de grande valia os detalhes apontados sobre o \textit{software} Asterisk, principalmente a conceituação e protocolos disponibilizados para comunicação com sistemas externos, contidos no próprio \textit{website} da \textit{Digium \footnote{Disponível em: \url{http://www.digium.com/}}}. 


\section{\textbf{TECNOLOGIAS UTILIZADAS}}

No estudo para propor uma solução viável e consistente de integração entre sistemas, que seja realmente eficiente, é necessário entender todo o contexto em que está sendo operado o Sistema de Informação GSAN, visando identificar as informações mais relevantes acessadas pelo atendimento ao cliente, as principais dificuldades enfrentadas e os desafios que norteiam esse módulo do sistema. 



\subsection{\textbf{\uppercase{Simple Object Access Protocol}}}
O protocolo Simple Object Access Protocol (SOAP) tem o objetivo de possibilitar a troca de informações estruturadas em Linguagem de Marcação Extensível (XML), para sistemas distribuídos. A negociação e transmissão de mensagens foram baseadas em outros em outros protocolos de serviços como o HTTP\footnote{HTTP - \textit{Hypertext Transfer Protocol}}  e RPC\footnote{RPC - \textit{Remote Procedure Call}}, possibilitando a utilização para realizar integrações entre softwares.

\subsection{\textbf{\uppercase{Asterisk Gateway Interface}}}
O software Asterisk possui uma interface de comunicação chamado AGI\footnote{AGI - \textit{Asterisk Gateway Interface}} \cite{asteriskAgi}, que tem como objetivo prover uma maior flexibilidade para adaptar soluções de linguagens diferentes, com esta interface se torna possível a comunicação com recursos externos através de requisição semelhantes ao CGI\footnote{CGI - \textit{Common Gateway Interface}} de servidores web, onde as requisições são originadas pelo próprio Asterisk, existem diversos \textit{frameworks} que implementam essa interface de comunicação, para este trabalho será utilizado o framework Asterisk-Java, por ser escrito sobre a plataforma Java e compatível com as tecnologias que foram utilizadas para a comunicação com o WebService do sistema GSAN.
O \textit{framework} Asterisk-Java possui diversos recursos disponíveis para comunicação com a ferramenta Asterisk, a seguir pode ser visto alguns dos principais propostos pelo framework;

\begin{itemize}
	\item Comunicação AGI - A classe BaseAgiScript.
	\item Comunicação HTTP - A classe ManagerConnection
	\item Ouvintes - As interfaces AsteriskServerListener e PropertyChangeListener.
	\item Controladores - A classes AsteriskServer e DefaultAsteriskServer
\end{itemize}


\subsection{\textbf{\uppercase{WebServices}}}
O WebService se trata de uma solução que permite que sistemas diferentes se comuniquem através requisições de protocolo HTTP a recursos identificados por um URI\footnote{URI - \textit{Uniform Resource Identifier}}  identifico a Web convencional, descritos e definidos usando XML\footnote{XML - \textit{Extensible Markup Language}} .

\subsection{\textbf{\uppercase{Unidade de Resposta Audível}}}
A Unidade de Resposta Audível\label{key:URA} (URA) ou atendente eletrônico se trata de um software ou equipamento de \textit{Call Center}, que possibilita o atendimento das ligações de forma automática, tal solução traz como benefício à padronização dos atendimentos e tem potencial para automatização dos atendimentos, com inúmeras possibilidades de customização através de integrações com sistemas externo \cite{VIEIRA:2007}.

\subsection{\textbf{\uppercase{Middleware}}}
O \textit{Middleware} ou intermediário se trata de uma camada de software responsável em mediar à comunicação de outros sistemas, utilizado normalmente em ambientes que tendem a utilizar plataformas, linguagens ou protocolos de comunicação diferentes nas trocas de informação, sendo um dos recursos adotados neste trabalho, tal recurso descrito por \citeonline{ALMEIDA:2011}.

\subsection{\textbf{\uppercase{Disc-OS}}}
O Disc-OS refere-se a uma distribuição Linux chamada Cent-OS, customizada para utilização de PABX e PABX IP, abstrai toda a configuração de bibliotecas básicas e instalação do Asterisk \cite{DARU:2008}. Disponibiliza uma interface Web para configuração dos principais recursos oque tornar bem prático o processo de configurações. Possui habilitado várias regras de \textit{firewall} pré-configurado por questões de segurança.

\subsection{\textbf{\uppercase{Codec}}}
O codec (COder/DEcoder) se trata de processo de codificação e decodificação da voz humana a ser transmitidas entre a origem e destino em meio digital \cite{VIEIRA:2007}.

\subsection{\textbf{\uppercase{JUnit Framework}}}
O JUnit framework destina-se a garantia da qualidade do software, viabilizando a construção dos mais diversos tipos de testes através do seu arcabouço de recursos disponíveis, por ser software livre e adotado como padrão nas principais IDE de desenvolvimento Java, ganhou grande popularidade nas comunidades de desenvolvedores. O desenvolvimento de testes unitários tem sido adotado como métricas de qualidade na entrega do produto de software.
O \textit{framework} tornou a escrita de testes um processo fácil e intuitivo, fazendo uso de recurso da plataforma Java chamado \textit{Annotation}\label{key:annotation}, trouxe a possibilidade de padronizar os métodos de testes apenas adicionando anotações sobres os mesmo, segue abaixo algumas anotações comumente utilizadas;

\begin{itemize}
	\item @Test - Anotação que representa um método de teste.
	\item @Before - Anotação indica que o método anotado será executado sempre antes de um método de teste.
 	\item @After - Anotação indica que o método anotado será executado sempre após a um método de teste.
\end{itemize}

O JUnit possui um objeto chamado \textit{Assert}, que contém uma série de validações possíveis para checagem do resultado esperado pelo teste.
