\chapter[Considerações Finais e Trabalhos Futuros]{\textbf{C}onsiderações Finais e \textbf{T}rabalhos Futuros}
%\addcontentsline{toc}{chapter}{Considerações Finais e Trabalhos Futuros}

A solução em \textit{software} apresentada e desenvolvida se trata de uma proposta inicial que pode se tornar um módulo interno com grandes potenciais para integrações futuras, neste trabalho foi adotada a integração do sistema GSAN com uma central de atendimento automatizada através do \textit{software} Asterisk, os novos serviços desenvolvidos possibilitam a reutilização para quaisquer outras soluções do gênero, o código fonte será disponibilizado na comunidade do GSAN, para que seja mantido e evoluído ao longo do tempo. Existem alguns recursos ou rotinas que podem ser criados para melhor aproveitamento e aprimoramento da solução proposta.

\begin{itemize}
	\item Reconhecimento de voz, para os casos de falta de água, que ocorre em regiões fora do endereço do cliente, de forma que o cliente possa informar a Rua que ocorre o problema.
	\item Automatizar serviços como negociação de débitos do cliente para serem realizados através da URA.
	\item Automatizar o \textit{feedback} dos serviços realizados, ou seja, a cada térmico de Ordem de Serviço o GSAN solicita do Asterisk que ele contate o cliente.
	\item Utilizar o Interceptador gerando estatísticas dos acessos realizados, coletando tempo de respostas e traçando perfis de cliente que mais realizam solicitações via \textit{Call Center}.
\end{itemize}


Estes são cenários que podem contribuir para o melhor aproveitamento da solução, com potencial de redução ainda maior dos registros de atendimentos. 
