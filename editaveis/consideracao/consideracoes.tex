\chapter[Considerações Finais e Trabalhos Futuros]{\textbf{C}onsiderações Finais e \textbf{T}rabalhos Futuros}
%\addcontentsline{toc}{chapter}{Considerações Finais e Trabalhos Futuros}

\section{Considerações Finais}
A solução em \textit{software} apresentada e desenvolvida trata-se de uma integração entre os sistemas GSAN e Asterisk, com o objetivo de automatizar os atendimentos destinados aos serviços Obter 2ª via de conta, Informar falta de água e Solicitar restabelecimento da ligação de água. Foi necessário desenvolver um \textit{Middleware} para intermediar a comunicação entre os sistemas e posteriormente uma suíte de testes automatizadas para garantir o funcionamento da solução. Dessa forma possibilitou automatizar o atendimento dos serviços descritos acima utilizando uma Unidade de Resposta Audível com um fluxo pré-determinado e consequentemente realizar a padronização do fluxo de atendimento.

Levando em consideração os resultados obtidos através da aplicação de todos os cenários de testes descritos no capitulo anterior, sob os serviços desenvolvidos neste trabalho, é possível afirmar que a utilização da Unidade de Resposta Audível na realização dos atendimentos de primeiro nível, tem potencial para reduzir parte dos atendimentos destinados aos serviços Obter 2ª via de conta, Informar falta de água e Solicitar restabelecimento da ligação de água, além de favorecer a redução dos custos.

Toda solução apresentada será adicionada na comunidade do sistema GSAN, para que seja mantida e evoluída ao longo do tempo, servindo como contribuição para os utilizadores do software.

\section{Trabalhos Futuros}

Existem alguns recursos ou rotinas que podem ser criados para melhor aproveitamento da solução proposta.

\begin{itemize}
	\item Reconhecimento de voz, para os casos de falta de água, que ocorre em regiões fora do endereço do cliente, de forma que o cliente possa informar a Rua que ocorre o problema.
	\item Automatizar serviços como negociação de débitos do cliente para serem realizados através da URA.
	\item Automatizar o \textit{feedback} dos serviços realizados, ou seja, a cada término de Ordem de Serviço o GSAN solicita do Asterisk que ele contate o cliente.
	\item Gerar estatísticas dos acessos realizados, coletando tempo de respostas e traçando perfis de cliente que mais realizam solicitações via \textit{Call Center}.
	\item Realizar testes de performance utilizando a ferramenta \textit{Sipp} para identificar a capacidade de resiliência da solução em situações adversas. 
	\item Implantar a solução em um ambiente real.
\end{itemize}


Estes são cenários que podem contribuir para um melhor aproveitamento da solução, com potencial de redução ainda maior dos registros de atendimentos. 
