% resumo em português
\setlength{\absparsep}{18pt} % ajusta o espaçamento dos parágrafos do resumo
\begin{resumo}
 
Este trabalho apresenta um dos principais sistemas utilizado para gestão de operações comerciais e controle de execução de serviços do setor de Saneamento Básico Brasileiro, o sistema GSAN, com melhorias no que diz respeito ao Atendimento ao Público, resultado da padronização dos atendimentos de primeiro nível e automatização dos atendimentos dos serviços Obter 2ª de Conta, Informar Falta de Água e Solicitar Restabelecimento da Ligação de Água, integrados a uma central de atendimento personalizada através da ferramenta Asterisk, que permite a utilização de voz sobre IP, além do uso convencional da telefonia pública como meio de comunicação com o cliente. Realizado para possibilitar a redução dos custos com atendimento ao cliente e pelo fato das empresas de saneamento serem altamente demandada pela população diariamente. Após o estudo aprofundado dos sistemas envolvidos, foi possível identificar uma forma de integrar as tecnologias de paradigmas diferentes, com a utilização de um agente intermediário responsável pela comunicação via protocolo SOAP e a interface de comunicação AGI, respectivamente para interligar os sistemas GSAN e Asterisk.  Foram realizados experimentos sobre o produto gerado, após a aplicação dos diversos cenários de testes foi demostrado uma automatização dos processos burocráticos no registro de solicitações que favorece a uma possível redução nos custos de atendimentos.


 \textbf{Palavras-chaves}: GSAN. Asterisk. Call Center.
\end{resumo}