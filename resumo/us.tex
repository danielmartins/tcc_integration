% resumo em inglês
\begin{resumo}[Abstract]
 \begin{otherlanguage*}{english}
This work presents one of the main systems used for managing business operations and execution control services of the Brazilian basic sanitation sector, GSAN system, with improvements with regard to the Public Service as a result of standardization of top-level visits and automation of care services second copy account, Inform Water Lack and Request Restoration of connection, integrated to a central personalized service through Asterisk tool, which allows the use of voice over IP in addition to the conventional use of public telephony as a means communication with the client. Carried out to enable the reduction of customer service costs and because the sanitation companies are highly requested by the population daily. After thorough study of the systems involved, it was possible to identify a way to integrate the different paradigms technologies, using an intermediary \textit{Middleware} responsible for communicating via SOAP protocols and a communication interface AGI respectively to interconnect GSAN and Asterisk systems. Experiments were carried out on the product generated, after the implementation of the various scenarios testing was demonstrated one automation of bureaucratic processes in the registration requests favoring a possible reduction in care costs.

   \vspace{\onelineskip}
 
   \noindent 
   \textbf{Key-words}: GSAN. Asterisk. Cell Center. Middleware. Brazilian basic sanitation.
 \end{otherlanguage*}
\end{resumo}